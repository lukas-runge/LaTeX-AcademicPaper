% -------------------- Spracheinstellungen --------------------

% Sprachpaket laden
\usepackage[ngerman]{babel}

% Umlaute
\usepackage[utf8]{inputenc} % Eingaben von Sonderzeichen codieren
\usepackage[T1]{fontenc} % deutsche Sonderzeichen als Schrift encodieren

% Bezeichnungen ändern
\addto{\captionsngerman}{%
  \renewcommand*{\contentsname}{Inhaltsverzeichnis} % Titel des Inhalsverzeichnisses ändern
  \renewcommand*{\listfigurename}{Abbildungsverzeichnis} % Titel des Abbildungsverzeichnisses ändern
  \renewcommand*{\listtablename}{Tabellen} % Titel des Tabellenverzeichnisses ändern
  \renewcommand*{\figurename}{Abb.} % Kürzel für Abbildungen festlegen
  \renewcommand*{\tablename}{Tab.} % Kürzel für Tabellen festlegen
  \renewcommand*{\abstractname}{Kurzreferat} % Titel für Zusammenfassungsseite (Abstract) festlegen
}


% -------------------- Seiteneinstellungen --------------------

% Seitenränder definieren
\newcommand{\sroben}{2.5cm} % Seitenrand oben
\newcommand{\srunten}{3cm} % Seitenrand unten
\newcommand{\srlinks}{2cm} % Seitenrand links
\newcommand{\srrechts}{2cm} % Seitenrand rechts

% Seitenränder anwenden, Header & Footer deaktivieren
\usepackage[\seitenformat,left=\srlinks,right=\srrechts,top=\sroben,bottom=\srunten,nohead,nofoot]{geometry}

% Automatische Silbentrennung aktivieren
\usepackage[document]{ragged2e}

% Seitenzahlen entfernen
\pagestyle{empty}

% Auflösung für Bilder in der *.pdf festlegen in DPI
\pdfimageresolution=300


% -------------------- Farben --------------------

% Faben definieren
\definecolor{blk}{HTML}{000000}
\definecolor{gr}{HTML}{494949}
\definecolor{bluish}{HTML}{0098A1}


% -------------------- Schrifteinstellungen --------------------

% Palatino als Schrift für das ganze Dokument festlegen
\usepackage{palatino}

% Schriftgröße der Überschriften festlegen
\newcommand{\fssec}{12} % Schriftgröße für Sektionsüberschrift 1
\newcommand{\fssubsec}{12} % Schriftgröße für Sektionsüberschrift 2
\newcommand{\fssubsubsec}{12} % Schriftgröße für Sektionsüberschrift 3

% Schriftgrößen auf Überschriften übernehmen
\sectionfont{\fontsize{\fssec}{15}\selectfont}
\subsectionfont{\fontsize{\fssubsec}{15}\selectfont}
\subsubsectionfont{\fontsize{\fssubsubsec}{15}\selectfont}

% Bilduntertitel formatieren
\usepackage[font=footnotesize, % Schriftgröße 10
            format=plain, 
            justification=centering, % zentrieren
            singlelinecheck=off, % Mehrzeiligkeit erlaubt
            labelfont={bf,it}, % Bez. Fett und Kursiv
            textfont=it]{caption} % Text nur Kursiv


% -------------------- Abstände --------------------

% Zeilenabstand festlegen
\setstretch{1,33}

% Abstände zwischen Titeln und Fließtext
\usepackage[compact]{titlesec}
%\titlespacing*{<command>}{<left>}{<before-sep>}{<after-sep>}
\titlespacing{\section}{0pt}{28pt}{10pt}
\titlespacing{\subsection}{0pt}{16pt}{*0}
\titlespacing{\subsubsection}{0pt}{*0}{*0}

% Abstände von Bilduntertiteln definieren
\setlength{\intextsep}{0pt}
\abovecaptionskip=4pt
\belowcaptionskip=2pt


% -------------------- Fußnoten --------------------

% Fußnoteneinzug rechts entfernen


% Seitenzahlen vollständig entfernen
%\pagenumbering{gobble}

% Verweise formatieren
\renewcommand{\footnoterule}{
                                \color{gr} % Farbe von Verweisetext festlegen
                                \textcolor{bluish}{\rule{6cm}{1pt}} % Farbe und Abmessungen des Trennstrichs festlegen
                            }
                            
% Abstände von Verweisen konfigurieren
\addtolength{\skip\footins}{6pt}
\let\oldfootnoterule\footnoterule
\def\footnoterule{\vskip-6pt\oldfootnoterule \vskip6pt\relax}
%\setlength{\footnotesep}{12pt}


% -------------------- Fußnoten --------------------

% Kurzzusammenfassung formatieren
\usepackage{abstract}
\renewcommand{\abstractnamefont}{\normalfont\normalsize\bfseries}
\renewcommand{\absnamepos}{flushleft}
\setlength{\absleftindent}{0pt}
\setlength{\absrightindent}{0pt}