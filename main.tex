% -------------------- Variablen definieren --------------------
\newcommand{\seitenformat}{a4paper} % Seitenformat festlegen
\newcommand{\schriftgroesse}{12pt} % Schriftgröße für Fließtext festlegen

\newcommand{\bildordner}{img} % Unterordner für Bilder festlegen


% -------------------- Dokument beginnen ---------------------
\documentclass[\schriftgroesse,\seitenformat]{article}


% -------------------- Importieren von Paketen --------------------

% Blindtextpakete laden
\usepackage{blindtext}
\usepackage{lipsum}

% Fußnote bändigen
\usepackage[hang,flushmargin]{footmisc} 

% Mehrzeilige Kommentare
\usepackage{verbatim}   

% Paket um einfach Zeilenabstände festzulegen
\usepackage{setspace}

% Paket um Bilder einbinden zu können
\usepackage{graphicx} % Bilder grundsätzlich einbinden können
\usepackage{wrapfig} % Bilder Textumfließend einbinden können

% Pakete zur einfacheren Formatanpassung
\usepackage{texilikechaps}
\usepackage{sectsty}

% Farbe - Yayyy
\usepackage{color}
\usepackage{xcolor}

% -------------------- Imporieren der selbsterstellten Funktionen --------------------
% Bild zum Einfügenen einer textumflossenden Grafik  

\newcommand{\warpfig}[5]{
\begin{wrapfigure}{#5}{#2\textwidth}
    \centering
    \vspace{-1pt}
    \includegraphics[width=#2\textwidth]{\bildordner/#1}
    \caption{#3}
    \label{fig:#4}
\end{wrapfigure}}

% --------------------



% Umbruch einfügen 

\newcommand{\br}{\leavevmode \newline}

% --------------------

% Bild zum Einfügenen einer Grafik

\newcommand{\Bild}[5]{
    \begin{figure}[htbp]
    \centering
    \vspace{#5pt}
     \includegraphics[width=#2\textwidth]{\bildordner/#1}
    \caption{#3}
    \label{fig:#4}
\end{figure}}

% --------------------



% Make greek possible

\newcommand{\textgreek}[1]{\begingroup\fontencoding{LGR}\selectfont#1\endgroup}

% --------------------


% -------------------- Importieren der Konfiguration --------------------
% Pfad für Bilder definieren
\graphicspath{\bildordner/}




% -------------------- Importieren der Formatvorgaben --------------------
% -------------------- Spracheinstellungen --------------------

% Sprachpaket laden
\usepackage[ngerman]{babel}

% Umlaute
\usepackage[utf8]{inputenc} % Eingaben von Sonderzeichen codieren
\usepackage[T1]{fontenc} % deutsche Sonderzeichen als Schrift encodieren

% Bezeichnungen ändern
\addto{\captionsngerman}{%
  \renewcommand*{\contentsname}{Inhaltsverzeichnis} % Titel des Inhalsverzeichnisses ändern
  \renewcommand*{\listfigurename}{Abbildungsverzeichnis} % Titel des Abbildungsverzeichnisses ändern
  \renewcommand*{\listtablename}{Tabellen} % Titel des Tabellenverzeichnisses ändern
  \renewcommand*{\figurename}{Abb.} % Kürzel für Abbildungen festlegen
  \renewcommand*{\tablename}{Tab.} % Kürzel für Tabellen festlegen
  \renewcommand*{\abstractname}{Kurzreferat} % Titel für Zusammenfassungsseite (Abstract) festlegen
}


% -------------------- Seiteneinstellungen --------------------

% Seitenränder definieren
\newcommand{\sroben}{2.5cm} % Seitenrand oben
\newcommand{\srunten}{3cm} % Seitenrand unten
\newcommand{\srlinks}{2cm} % Seitenrand links
\newcommand{\srrechts}{2cm} % Seitenrand rechts

% Seitenränder anwenden, Header & Footer deaktivieren
\usepackage[\seitenformat,left=\srlinks,right=\srrechts,top=\sroben,bottom=\srunten,nohead,nofoot]{geometry}

% Automatische Silbentrennung aktivieren
\usepackage[document]{ragged2e}

% Seitenzahlen entfernen
\pagestyle{empty}

% Auflösung für Bilder in der *.pdf festlegen in DPI
\pdfimageresolution=300


% -------------------- Farben --------------------

% Faben definieren
\definecolor{blk}{HTML}{000000}
\definecolor{gr}{HTML}{494949}
\definecolor{bluish}{HTML}{0098A1}


% -------------------- Schrifteinstellungen --------------------

% Palatino als Schrift für das ganze Dokument festlegen
\usepackage{palatino}

% Schriftgröße der Überschriften festlegen
\newcommand{\fssec}{12} % Schriftgröße für Sektionsüberschrift 1
\newcommand{\fssubsec}{12} % Schriftgröße für Sektionsüberschrift 2
\newcommand{\fssubsubsec}{12} % Schriftgröße für Sektionsüberschrift 3

% Schriftgrößen auf Überschriften übernehmen
\sectionfont{\fontsize{\fssec}{15}\selectfont}
\subsectionfont{\fontsize{\fssubsec}{15}\selectfont}
\subsubsectionfont{\fontsize{\fssubsubsec}{15}\selectfont}

% Bilduntertitel formatieren
\usepackage[font=footnotesize, % Schriftgröße 10
            format=plain, 
            justification=centering, % zentrieren
            singlelinecheck=off, % Mehrzeiligkeit erlaubt
            labelfont={bf,it}, % Bez. Fett und Kursiv
            textfont=it]{caption} % Text nur Kursiv


% -------------------- Abstände --------------------

% Zeilenabstand festlegen
\setstretch{1,33}

% Abstände zwischen Titeln und Fließtext
\usepackage[compact]{titlesec}
%\titlespacing*{<command>}{<left>}{<before-sep>}{<after-sep>}
\titlespacing{\section}{0pt}{28pt}{10pt}
\titlespacing{\subsection}{0pt}{16pt}{*0}
\titlespacing{\subsubsection}{0pt}{*0}{*0}

% Abstände von Bilduntertiteln definieren
\setlength{\intextsep}{0pt}
\abovecaptionskip=4pt
\belowcaptionskip=2pt


% -------------------- Fußnoten --------------------

% Fußnoteneinzug rechts entfernen


% Seitenzahlen vollständig entfernen
%\pagenumbering{gobble}

% Verweise formatieren
\renewcommand{\footnoterule}{
                                \color{gr} % Farbe von Verweisetext festlegen
                                \textcolor{bluish}{\rule{6cm}{1pt}} % Farbe und Abmessungen des Trennstrichs festlegen
                            }
                            
% Abstände von Verweisen konfigurieren
\addtolength{\skip\footins}{6pt}
\let\oldfootnoterule\footnoterule
\def\footnoterule{\vskip-6pt\oldfootnoterule \vskip6pt\relax}
%\setlength{\footnotesep}{12pt}


% -------------------- Fußnoten --------------------

% Kurzzusammenfassung formatieren
\usepackage{abstract}
\renewcommand{\abstractnamefont}{\normalfont\normalsize\bfseries}
\renewcommand{\absnamepos}{flushleft}
\setlength{\absleftindent}{0pt}
\setlength{\absrightindent}{0pt}


% -------------------- Importieren des Glossars --------------------
% --- Glossar ---
\usepackage{glossaries} % Paket importieren


\newglossaryentry{moment} % Eintrag 1
{
  name={Moment}, % Begriff 1
  description={Bitcoin is a peer-to-peer payment system and digital currency introduced as open source software in 2009 by pseudonymous developer Satoshi Nakamoto. It is a cryptocurrency, so-called because it uses cryptography to control the creation and transfer of money} % Beschreibung 1
}


\newglossaryentry{mannschaft} % Eintrag 2
{
  name={Mannschaft}, % Begriff 2
  description={ist ein deutscher Dokumentarfilm über die Fußball-Weltmeisterschaft 2014 in Brasilien, der sie aus Sicht der deutschen Mannschaft zeigt. Er wurde von den DFB-Mitarbeitern Martin Christ, Jens Gronheid und Ulrich Voigt erstellt} % Beschreibung 2
}

% Glossar Datei erstellen
\makenoidxglossaries


% -------------------- Festlegen der Zitierweise --------------------
\usepackage{hyperref}
\usepackage{csquotes}
\usepackage[backend=biber, sorting=nyt]{biblatex}
\bibliography{bib/literatur}

% -------------------- Inhaltlicher Beginn des Dokuments --------------------
\begin{document}

% Importiern des wiss. Inhalts

%Titelseite
\begin{titlepage}
	\centering
	\includegraphics[width=0.15\textwidth]{img/beuth.png}\par\vspace{1cm}
	{\scshape\LARGE Beuth Hochschule für Technik Berlin \par}
	\vspace{1cm}
	{\scshape\Large wissenschaftliche Arbeit\par}
	\vspace{1.5cm}
	{\huge\bfseries [Hier könnte ihr Titel stehen]\par}
	\vspace{2cm}
	{\Large\itshape Max Mustermann\\Olaf Schulze\\Svenja Meier\par}
	\vfill
	betreut von\par
	Prof. MA Dipl.-Ing. Tina \textsc{Kitzing}

	\vfill
	
	{\large \today\par} % Datum
\end{titlepage}

% Kurzreferat
\begin{abstract} \noindent\RaggedRight\normalsize
Diese Dokumentation enthält eine sortierte Liste der wichtigsten
\LaTeX--Befehle. Die einzelnen Listeneintr"age sind untereinander
durch viele Querverweise verkettet, die ein Auffinden inhaltlich
zusammengeöriger Informationen erheblich erleichtern.
\end{abstract}

% Glossar
\clearpage
\printnoidxglossaries

% Inhaltsverzeichnis
\clearpage
\tableofcontents
\clearpage

%Inhalt
\section{Rede Giovanni Trapattoni '98} % Überschrift erster Ordnung

\subsection{Was erlauben Strunz?} % Überschrift zweiter Ordnung


\warpfig{testbild.jpg}{0.25}{Demonstrierender Student}{gilbert}{l}

Es gibt im \gls{moment} in diese \gls{mannschaft}, oh, einige Spieler\footnote{Hallo} vergessen ihnen Profi was sie sind. Ich lese nicht sehr viele Zeitungen, aber ich habe gehört viele Situationen. Erstens: wir haben nicht offensiv gespielt. Es gibt keine deutsche Mannschaft spielt offensiv und die Name offensiv wie Bayern. Letzte Spiel hatten wir in Platz drei Spitzen: Elber, Jancka und dann Zickler. Wir müssen nicht vergessen Zickler. Zickler ist eine Spitzen mehr, Mehmet eh mehr Basler. Ist klar diese Wörter, ist möglich verstehen, was ich hab gesagt? Danke. Offensiv, offensiv ist wie machen wir in Platz. Zweitens: ich habe erklärt mit diese zwei Spieler: nach Dortmund brauchen vielleicht Halbzeit Pause. Ich habe auch andere Mannschaften gesehen in Europa nach diese Mittwoch. Ich habe gesehen auch zwei Tage die Training. Ein Trainer ist nicht ein Idiot! Ein Trainer sei sehen was passieren in Platz. In diese Spiel es waren zwei, drei diese Spieler waren schwach wie eine Flasche leer! Haben Sie gesehen Mittwoch, welche Mannschaft hat gespielt Mittwoch? Hat gespielt Mehmet oder gespielt Basler oder hat gespielt Trapattoni? Diese Spieler beklagen mehr als sie spielen! Wissen Sie, warum die Italienmannschaften\footnote{Ich bin auch noch eine Fußnote, mich beachtet doch ehh niemand - deshalb bin ich jetzt besonders lange, einfach weil ichs kann!} kaufen nicht diese Spieler? Weil wir haben gesehen viele Male solche Spiel! Haben gesagt sind nicht Spieler für die italienisch Meisters! Strunz! Strunz ist zwei Jahre hier, hat gespielt 10 Spiele, ist immer verletzt! Was erlauben Strunz? Letzte Jahre Meister Geworden mit Hamann, eh, Nerlinger. Diese Spieler waren Spieler! Waren Meister geworden! Ist immer verletzt! Hat gespielt 25 Spiele in diese Mannschaft in diese Verein. Muß respektieren die andere Kollegen! haben viel nette kollegen! Stellen Sie die Kollegen die Frage! Haben keine Mut an Worten, aber ich weiß, was denken über diese Spieler. Mussen zeigen jetzt, ich will, Samstag, diese Spieler müssen zeigen mich, seine Fans, müssen alleine die Spiel gewinnen. Muß allein die Spiel gewinnen! Ich bin müde jetzt Vater diese Spieler, eh der Verteidiger diese Spieler. Ich habe immer die Schuld über diese Spieler. Einer ist Mario, einer andere ist Mehmet! Strunz ich spreche nicht, hat gespielt nur 25 Prozent der Spiel. Ich habe fertig! \footnote{Spoiler: Hat er nicht... :(}
\clearpage

\subsection{Obwohl er schon fertig hatte}
Wenn es gab Fragen, ich Worte wiederholen... Es gibt im Moment in diese Mannschaft, oh, einige Spieler vergessen ihnen Profi was sie sind. Ich lese nicht sehr viele Zeitungen, aber habe gehört viele Situationen. Erstens: wir haben nicht offensiv gespielt. Es gibt keine deutsche Mannschaft spielt offensiv und die Name offensiv wie Bayern. Letzte Spiel hatten wir in Platz drei Spitzen: Elber, Jancka und dann Zickler. Wir müssen nicht vergessen Zickler. Zickler ist eine Spitzen mehr, Mehmet eh mehr Basler. Ist klar diese Wörter, ist möglich verstehen, was ich gesagt? Danke. Offensiv, offensiv ist wie machen in Platz. Zweitens: ich habe erklärt mit diese zwei Spieler: nach Dortmund brauchen vielleicht Halbzeit Pause. Ich habe auch andere Mannschaften gesehen in Europa nach diese Mittwoch. Ich habe gesehen zwei Tage die Training. Ein Trainer ist nicht ein Idiot! Ein Trainer sei sehen was passieren in Platz. In diese Spiel es waren zwei, drei diese Spieler waren schwach wie eine Flasche leer! Haben Sie gesehen Mittwoch, welche Mannschaft hat gespielt Mittwoch? Hat gespielt Mehmet oder gespielt Basler oder hat gespielt Trapattoni? Diese Spieler beklagen mehr als sie spielen! Wissen Sie, warum die Italienmannschaften diese Spieler nicht kauften? Weil wir haben gesehen viele Male solche Spiel! Haben gesagt sind nicht Spieler für die italienisch Meister! Strunz! Strunz ist zwei Jahre hier, hat gespielt 100 Spiele, ist immer verletzt! Was erlauben Strunz? Letzte Jahre Meister Geworden mit Hamann, eh, Nerlinger. Diese Spieler waren Spieler! Waren Meister geworden! Ist immer verletzt! Hat gespielt 25 Spiele in diese Mannschaft in diese Verein. Muß respektieren die andere Kollegen! haben viel nette kollegen! Stellen Sie die Kollegen die Frage! Haben keine Mut an Worten, aber ich weiß, was denken über diese Spieler. Mussen zeigen jetzt, ich will, Samstag, diese Spieler müssen zeigen mich, seine Fans , müssen alleine die Spiel gewinnen. Muß allein die Spiel gewinnen! Ich bin müde jetzt Vater diese Spieler, eh der Verteidiger diese Spieler. Ich habe immer die Schuld über diese Spieler. Einer ist Mario, einer andere ist Mehmet! Strunz ich spresche nicht, hat gespielt nur 25 Prozent der Spiel. Ich habe fertig! ...wenn es gab Fragen, ich kann Worte wiederholen...Es gibt im Moment in diese Mannschaft, oh, einige Spieler vergessen ihnen Profi was sie sind. Ich lese nicht sehr viele Zeitungen, aber ich habe gehört viele Situationen. Erstens: wir haben nicht offensiv gespielt. Es gibt keine deutsche Mannschaft spielt offensiv und die Name offensiv wie Bayern. Letzte Spiel hatten wir in Platz drei Spitzen: Elber, Jancka und dann Zickler. Wir müssen nicht vergessen Zickler.
\section{Obwohl er schon fertig hatte}
\subsection{Blindtext}
Zickler \footfullcite[vgl.][22]{l_1} ist eine Spitzen mehr, Mehmet eh mehr Basler. Ist klar diese Wörter, ist möglich verstehen, was ich hab gesagt? Danke. Offensiv, offensiv ist wie machen wir in Platz. Zweitens: ich habe erklärt mit diese zwei Spieler: nach Dortmund brauchen vielleicht Halbzeit Pause. Ich habe auch andere Mannschaften gesehen in Europa nach diese Mittwoch. Ich habe gesehen auch zwei Tage die Training.\footfullcite[vgl.][]{a_1} Ein Trainer ist nicht ein Idiot! Ein Trainer sei sehen was passieren in Platz. In diese Spiel es waren zwei, drei diese Spieler waren schwach wie eine Flasche leer! Haben Sie gesehen Mittwoch, welche Mannschaft hat gespielt Mittwoch? Hat gespielt Mehmet \footfullcite[vgl.][]{q_1} oder gespielt Basler oder hat gespielt Trapattoni?
\Bild{werbung.jpg}{1}{Werbung}{werbung}{6}
Diese Spieler beklagen mehr als sie spielen! Wissen Sie, warum die Italienmannschaften kaufen nicht diese Spieler? Weil wir haben gesehen viele Male solche Spiel! Haben gesagt sind nicht Spieler für die italienisch Meisters! Strunz! Strunz ist zwei Jahre hier, hat gespielt 10 Spiele, ist immer verletzt! Was erlauben Strunz? Letzte Jahre Meister Geworden mit Hamann, eh, Nerlinger. Diese Spieler waren Spieler! Waren Meister geworden! Ist immer verletzt! Hat gespielt 25 Spiele in diese Mannschaft in diese Verein. Muß\footnote{Rechtschreibungsfehler \frqq\,Bitte neue Rechtschreibung verwenden.} respektieren die andere Kollegen! haben\footnote{Groß-/Kleinschreibung \frqq\,Gehen sie bitte zurück in die Grundschule!} viel nette kollegen! Stellen Sie die Kollegen die Frage!

\clearpage
\printbibliography[heading=subbibliography,title={Literaturverzeichnis},nottype=online]

\clearpage
\printbibliography[heading=subbibliography,title={Quellenverzeichnis},type=online]

% Abbildungsverzeichnis
\clearpage
{%
\let\oldnumberline\numberline%
\renewcommand{\numberline}{\figurename~\oldnumberline}%
\listoffigures%
}



\end{document}
